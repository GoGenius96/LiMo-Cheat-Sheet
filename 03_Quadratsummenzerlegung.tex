\section*{Kapitel 3 - Quadratsummenzerlegung und statistische Inferenz im multiplen linearen Regressionsmodell}

\begin{multicols*}{3}

\tikzstyle{mybox} = [draw=black, fill=white, very thick,
    rectangle, rounded corners, inner sep=10pt, inner ysep=10pt]
\tikzstyle{fancytitle} =[fill=black, text=white, font=\bfseries]



%------------ Quadratsummenzerlegung ---------------
\begin{tikzpicture}
    \node [mybox] (box){%
        \begin{minipage}{0.3\textwidth}
        Gegeben sei das multiple lineare Regressionsmodell mit
        $\rang(\bX) = p'$. Dann gilt
        \footnotesize{
        $$
        \underbrace{(\bY - \bar\bY)^\top (\bY - \bar\bY)}_{SST} =
        \underbrace{(\bY - \hbY)^\top (\bY - \hbY)}_{SSE} + 
        \underbrace{(\hbY - \bar\bY)^\top (\hbY - \bar\bY)}_{SSM}.
        $$}

        \begin{align*}
        & \text{SST(otal):} & & \text{Gesamt-Quadratsumme (korrigiert)}\\
        & \text{SSE(rror):} & &\text{Fehler-Quadratsumme}\\
        & \text{SSM(odel):} & &\text{Modell-Quadratsumme}\\
        \end{align*}
        \end{minipage}
    };
%------------ Quadratsummenzerlegung Header ---------------------
\node[fill = purple, text=white, font=\bfseries, right=10pt] at (box.north west) {Quadratsummenzerlegung};
\end{tikzpicture}


%------------ Quadratsummenzerlegung ohne b0 ---------------
\begin{tikzpicture}
    \node [mybox] (box){%
        \begin{minipage}{0.3\textwidth}
        Gegeben sei das multiple lineare Regressionsmodell mit, aber ohne Absolutglied $\gb0$. Dann gilt
        \footnotesize{
        $$
        \underbrace{\bY^\top \bY}_{SST^*} =
        \underbrace{(\bY - \hbY)^\top (\bY - \hbY)}_{SSE} + 
        \underbrace{\hbY^\top \hbY}_{SSM^*}.
        $$}

        \begin{align*}
        & \text{$SST^*$:} & & \text{Gesamt-Quadratsumme (nicht korrigiert)}\\
        & \text{$SSE$:} & &\text{Fehler-Quadratsumme (wie zuvor)}\\
        & \text{$SSM^*$:} & &\text{Modell-Quadratsumme (nicht korrigiert)}\\
        \end{align*}
        \end{minipage}
    };
%------------ Quadratsummenzerlegung ohne b0 Header ---------------------
\node[fill = purple, text=white, font=\bfseries, right=10pt] at (box.north west) 
{Quadratsummenzerlegung ohne $\beta_0$};
\end{tikzpicture}


%------------ Erwartungswerte der Quadratsummen ---------------
\begin{tikzpicture}
    \node [mybox] (box){%
        \begin{minipage}{0.3\textwidth}
        Gegeben sei das multiple lineare Regressionsmodell mit den üblichen Annahmen. 
        Wir definieren
        $$\be = \begin{pmatrix}
            1 \\ \vdots \\ 1
        \end{pmatrix}  \text{ und } \bP_{\be} = \be(\be^\top \be)^{-1}\be^\top \text{ und } \bQ_{\be} = \bI - \bP_{\be}.$$
        Dann gilt
        \begin{align*}
        & \E(SST^*) &&= \ssd n + \bbeta^\top \bX^\top \bX \bbeta \\
        & \E(SST) &&= \ssd (n-1) + \bbeta^\top (\bQ_{\be}\bX)^\top (\bQ_{\be}\bX) \bbeta \\
        & \E(SSE) &&= \ssd (n-p') \\
        & \E(SSM^*) &&= \ssd p' + \bbeta^\top \bX^\top \bX \bbeta \\
        & \E(SSM^*) &&= \ssd (p'-1) + \bbeta^\top (\bQ_{\be}\bX)^\top (\bQ_{\be}\bX) \bbeta
        \end{align*}
        Wir können diese Eigenschaften zur Konstruktion von Tests verwenden.
        Es gilt nämlich unter anderem
        $\bbeta = 0 \implies \E(SST^*) = \ssd n$\\
        $\gb{1} = \dots = \gb{p} = 0 \implies \E(SSM) = \ssd (p'-1)$
        \end{minipage}
    };
%------------ Erwartungswerte der Quadratsummen Header ---------------------
\node[fill = purple, text=white, font=\bfseries, right=10pt] at (box.north west) 
{Erwartungswerte der Quadratsummen};
\end{tikzpicture}
    

%------------ Chi-Quadrat Verteilung ---------------
\begin{tikzpicture}
    \node [mybox] (box){%
        \begin{minipage}{0.3\textwidth}
        Sei $\bZ \sim \Ncal_n(\bmu, \bI)$, so heißt $\bW = \bZ^\top\bZ = \sumin Z_i^2$ (nicht-zentral) \tc{Chi-Quadrat-verteilt} und wir schreiben
        $$W \sim \chi^2(n, \delta).$$
        Wir nennen $n$ die \tc{Zahl der Freiheitsgrade} und $\delta = \bmu^\top\bmu$ den \tc{Nicht-Zentralitätsparameter}. 
        Es gilt
        \begin{align*}
            \E(W) &= n + \delta \\
            \V(W) &= 2n + 4\delta \\
        \end{align*}

        \end{minipage}
    };
%------------ Chi-Quadrat Verteilung Header ---------------------
\node[fill = black, text=white, font=\bfseries, right=10pt] at (box.north west) 
{Chi-Quadrat Verteilung};
\end{tikzpicture}

%------------ t-Verteilung ---------------
\begin{tikzpicture}
    \node [mybox] (box){%
        \begin{minipage}{0.3\textwidth}
        Seien $Z \sim \Ncal(\delta, 1)$ und $W \sim \chi^2(n, 0)$ unabhängig.
        Dann heißt $T = \frac{Z}{\sqrt{\frac{W}{n}}}$ (nicht-zentral) \tc{t-verteilt} mit $n$ \tc{Freiheitsgraden} und \tc{Nicht-Zentralitätsparameter} $\delta$ 
        und wir schreiben $$T \sim t(n, \delta).$$
        Es gilt
        $$\E(T) = \delta \sqrt{\frac{n}{2}} \frac{\Gamma(\frac{n-1}{2})}{\Gamma(\frac{n}{2})} \text{ für } n > 1$$
        \end{minipage}
    };
%------------ t-Verteilung Header ---------------------
\node[fill = black, text=white, font=\bfseries, right=10pt] at (box.north west) 
{t-Verteilung};
\end{tikzpicture}

%------------ F-Verteilung ---------------
\begin{tikzpicture}
    \node [mybox] (box){%
        \begin{minipage}{0.3\textwidth}
        Sei $W_1 \sim \chi^2(n_1, \delta)$ und $W_2 \sim \chi^2(n_2, 0)$ unabhängig.
        Dann heißt $X =  \frac{W_1/n_1}{W_2/n_2}$ (nicht-zentral) \tc{F-verteilt} mit $n_1$ und $n_2$ \tc{Freiheitsgraden} und \tc{Nicht-Zentralitätsparameter} $\delta$
        und wir schreiben $$X \sim F(n_1, n_2, \delta).$$
        Es gilt
        $$\E(X) = \frac{n_2 + \frac{n_2\delta}{n_1}}{n_2 - 2} \text{ für } n_2 > 2$$
        \end{minipage}
    };
%------------ F-Verteilung Header ---------------------
\node[fill = black, text=white, font=\bfseries, right=10pt] at (box.north west) 
{F-Verteilung};
\end{tikzpicture}


\end{multicols*}