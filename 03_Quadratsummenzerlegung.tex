\section*{Kapitel 3 - Quadratsummenzerlegung und statistische Inferenz im multiplen linearen Regressionsmodell}

\begin{multicols*}{3}

\tikzstyle{mybox} = [draw=black, fill=white, very thick,
    rectangle, rounded corners, inner sep=10pt, inner ysep=10pt]
\tikzstyle{fancytitle} =[fill=black, text=white, font=\bfseries]



%------------ Quadratsummenzerlegung ---------------
\begin{tikzpicture}
    \node [mybox] (box){%
        \begin{minipage}{0.3\textwidth}
        Gegeben sei das multiple lineare Regressionsmodell mit
        $\rang(\bX) = p'$. Dann gilt
        \footnotesize{
        $$
        \underbrace{(\bY - \bar\bY)^\top (\bY - \bar\bY)}_{SST} =
        \underbrace{(\bY - \hbY)^\top (\bY - \hbY)}_{SSE} + 
        \underbrace{(\hbY - \bar\bY)^\top (\hbY - \bar\bY)}_{SSM}.
        $$}

        \begin{align*}
        & \text{SST(otal):} & & \text{Gesamt-Quadratsumme (korrigiert)}\\
        & \text{SSE(rror):} & &\text{Fehler-Quadratsumme}\\
        & \text{SSM(odel):} & &\text{Modell-Quadratsumme}\\
        \end{align*}
        \end{minipage}
    };
%------------ Quadratsummenzerlegung Header ---------------------
\node[fill = purple, text=white, font=\bfseries, right=10pt] at (box.north west) {Quadratsummenzerlegung};
\end{tikzpicture}
    
\end{multicols*}